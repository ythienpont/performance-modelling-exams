\exam{2022}
\begin{problem}{Poisson Process}
Consider a Poisson process with arrival rate $\lambda$ = 2. Assume there is one arrival in [0, 2] and there is one (possibly the same) arrival in [1, 3]. What is the probability that we have only one arrival in [0, 3]?
\end{problem}

\begin{solution}
We are looking for the following probability:
\begin{align*}
    &P(N_{0,3}=1 |N_{0,2}=1, N_{1,3}=1) 
\end{align*}
This is the same as looking for these three conditions:
\begin{enumerate}
    \item $P(N_{0,1}=0)$
    \item $P(N_{1,2}=1)$
    \item $P(N_{2,3}=0)$
\end{enumerate}
For each we can use the Poisson distribution , given us the following solutions:
\begin{align*}
    &P(N_{0,1}=0)= \frac{(\lambda t) ^0 e^ {-\lambda t}}{0!}= e^{-2}\\
    &P(N_{1,2}=1)=2.e^{-2}\\
    &P(N_{2,3}=0)=e^{-2}
\end{align*}
Since each probability (in each internal) is independent from each other , we can simplify mutliply all of them with each other giving us the final solution of $2e^{-6}$.
\end{solution}

\begin{problem}{Markov Chains}
Consider a DTMC on the state space \( S = \{(i_1, j_1, i_2, j_2) | 1 \leq i_1, j_1, i_2, j_2 \leq 8\} \), that is, the state of the DTMC is characterized by marking two squares \( (i_1, j_1) \) and \( (i_2, j_2) \) on an \( 8 \times 8 \) board. Assume the transition probabilities are such that the state \( (i'_1, j'_1, i'_2, j'_2) \) visited from state \( (i_1, j_1, i_2, j_2) \) is such that \( (i'_1, j'_1) \) is a random neighbor of \( (i_1, j_1) \) and \( (i'_2, j'_2) \) is a random neighbor of \( (i_2, j_2) \). For instance, from state \( (1, 3, 4, 4) \) we move to \( (1, 4, 4, 5) \) with probability \( \frac{1}{12} = \frac{1}{3} \cdot \frac{1}{4} \) as \( (1, 3) \) has 3 neighbors and square \( (4, 4) \) has 4 neighbors. How many communicating classes does this DTMC have? What is the period of each class?
\end{problem}

\begin{solution}
  From each square it is possible to reach any other square in a finite number of steps (irreducible). Since we can return to a square in an arbitrary number of steps, it is also aperiodic. Thus, we have a \textit{single} communicating class with a period $d=1$.
\end{solution}

\begin{problem}{Markov Chains}
Consider an irreducible, positive recurrent DTMC. Let \( P \) be its transition probability matrix and \( \pi = (\pi_0, \pi_1, \ldots) \) its unique stationary distribution. Give an expression for
\[
\tilde{p}_{i,j} = \lim_{n \to \infty} P[X_n = j | X_{n+1} = i],
\]
in terms of the entries of \( P \) and \( \pi \). Define a DTMC with transition probability matrix \( \tilde{P} \) such that entry \( (i, j) \) of \( \tilde{P} \) equals \( \tilde{p}_{i,j} \). Is this DTMC positive recurrent? If so, what can you say about its stationary distribution?
\end{problem}

\begin{solution}
  Using Bayes' theorem we can write $\tilde{p}_{i,j}$ as
  \[
    \tilde{p}_{i,j} = \lim_{n \to \infty} P[X_n = j | X_{n+1} = i] = \lim_{n \to \infty} \frac{P[X_{n+1}=i | X_n = j] \cdot P[X_n=j]}{P[X_{n+1}=i]}
  \]

  Since the process is stationary, $P[X_n=j] = \pi_j$ and $P[X_{n+1}=i] = \pi_i$. Also, $P[X_{n+1}=i | X_n = j] = P_{j,i}$, where $P_{j,i}$ is the transition probability from $j$ to $i$. Thus,
  \[
    \tilde{p}_{i,j} = \frac{P_{j,i}\cdot \pi_j}{\pi_i}
  \]

  We can now define the transition matrix $\tilde{P}$ where 
  \[
    \tilde{P}_{i,j}=\tilde{p}_{i,j} = \frac{P_{j,i}\cdot \pi_j}{\pi_i}
  \]

  If the original DTMC was irreducible, the reversed DTMC based on it's stationary distribution will also be irreducible, therefore positive recurrent.
  We know that $\tilde{\pi}$ must satisfy $\tilde{\pi}\tilde{P} = \tilde{\pi}$. For a fixed $j$, we find

  \[
    \tilde{\pi}_j = \sum_i\pi_i\cdot\tilde{p}_{i,j} = \sum_i\pi_i\frac{p_{j,i}\cdot \pi_j}{\pi_i}=\sum_i\pi_j\cdot  p_{j,i}= \pi_j\sum_ip_{j,i} = \pi_j \cdot 1 = \pi_j
  \]

  Thus,
  \[
    \boxed{\tilde{\pi} = \pi}
  \]
\end{solution}

\begin{problem}{Markov Chains}
Give an example of an irreducible, transient DTMC with period 2. Explain why your example has these properties.
\end{problem}

\begin{solution}
	Define the Markov chain with $S=\{1,2\}$ and a transition matrix $P$:
	\[
		P = \begin{pmatrix}
		0 & 1 \\
		1 & 0 \\
		\end{pmatrix}
	\]

	\begin{itemize}
		\item \textbf{Irreducible}: Each state can be reached from the other. Single communicating class.
		\item \textbf{Period of 2}: For each state, any (simple) path returning to it involves exactly two steps. (Every timestep we visit state $i$ after starting in $i$ is divisible by two).
		\item \textbf{Transience}: The chain never returns to the same state within one step.
	\end{itemize}
\end{solution}

\begin{problem}{PASTA}
Consider a queueing system with Poisson arrivals, where the service time of customer \( n \) depends on the inter-arrival time between customer \( n - 1 \) and \( n \). Can we apply the PASTA property for this queueing system? Explain your answer.
\end{problem}

\begin{problem}{Applications}
Consider a queueing network consisting of 2 queues (not necessarily M/M/1 queues). Type 1 jobs arrive at rate \( \lambda_1 \) and first visit queue 1 followed by queue 2. Type 2 jobs arrive at rate \( \lambda_2 \) and visit queue 1 only, while type 3 jobs arrive at rate \( \lambda_3 \) and only visit queue 2. Assume that the mean time that a random job spends in the queueing network equals 5, the mean time that a random type 1 or type 2 job spends in queue 1 is 3 and that the mean time that a random type 1 or 3 job spends in queue 2 is 4. Determine the arrival rates \( \lambda_1, \lambda_2 \) and \( \lambda_3 \) given that \( \lambda_1 = \lambda_2 \) and \( \lambda = \lambda_1 + \lambda_2 + \lambda_3 = 1 \). Does your answer change if type 1 jobs visit both queues in the opposite order? Explain.
\end{problem}

\begin{problem}{Applications}
Consider an M/M/1 queue (arrival rate \( \lambda \), service rate \( \mu \)) with the additional property that when a customer arrives it starts an exponential timer with mean \( 1/\theta \) and leaves the queueing system immediately if that timer expires before its service starts. How can you model this queueing system using a CTMC? Give an expression for the stationary distribution. For which value of \( \theta \) is this distribution a Poisson distribution?
\end{problem}
