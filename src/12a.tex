\exam{2012a}
\begin{problem}{Bernoulli and Poisson process}
Indicate whether the Poisson process is a suitable process to model:
\begin{itemize}
\item[(a)] The time epochs at which hand-overs take place in a wireless network.
\item[(b)] The time epochs at which a packet is sent by a single TCP source.
\item[(c)] The time epochs at which packets are sent by a large number of TCP sources sharing a single link.
\item[(d)] The time epochs at which emails arrive in your mailbox.
\end{itemize}
\end{problem}

\begin{problem}{Branching Processes}
Give an example of a single-type branching process such that
the extinction probability is equal to $p$, for any $0 < p < 1$.
\end{problem}

\begin{problem}{Branching Processes}
Construct a branching process to model the coordinated splitting tree algorithm (CSTA). Whenever $n$ users collide in the CSTA, they are allowed to exchange information such that they split into $n$ groups, each consisting of a single user. Collisions still occur on the channel as colliding users get no information with respect to the arrival times of the new users. Explain how we can determine the maximum stable throughput by means of this branching process.
\end{problem}

\begin{problem}{Markov Chains}
Given an example of a Markov chain that is:
\begin{itemize}
    \item[(a)] Irreducible, positive recurrent with period $d = 3$.
    \item[(b)] Irreducible, transient, with period $d = 2$.
\end{itemize}
Explain in detail why your examples meet the above requirements.
\end{problem}

\begin{problem}{Markov Chains}
Assume we have 2 machines to process jobs. Machine $i$ needs to process $k_i$ jobs, and the processing time of a job on machine $i$ is exponential with mean $1/\mu_i$, respectively, for $i = 1$ and $2$. How can we determine the probability that machine 1 finishes its $k_1$ jobs before machine 2 finishes its $k_2$ jobs (use a Markov chain)? Assume that as soon as one machine finishes its work, it also starts to process pending jobs from the other machine (if any). How can we determine the mean time until all the $k_1 + k_2$ jobs are completed?
\end{problem}

\begin{problem}{Processor Sharing Queue}
Assume jobs arrive according to a Poisson process with rate $\lambda$ at a single server. A job requires an exponential amount of work with mean $1/\mu$. All the jobs in the queue share the single server, meaning a job is served at rate $\mu/n$ whenever there are $n$ jobs in the queue. Set up a Markov chain to determine the number of jobs in the queue (at arrival or departure times). Determine the condition for positive recurrence and give an explicit expression for the steady state probabilities. Give an expression for the queue length distribution at arrival times only. Do you recognize the latter expression from an earlier course?
\end{problem}

\begin{problem}{Bianchi Model}
Assume we use the simple ALOHA scheme to retransmit a packet instead of the binary exponential back-off algorithm, with the additional requirement that a packet can be retransmitted at most $m > 0$ times. Discuss the changes required to the Bianchi model to determine the saturation throughput. How does $m$ influence the saturation throughput?
\end{problem}
