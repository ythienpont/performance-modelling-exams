\exam{2015a}
\begin{problem}{Bernoulli and Poisson process} 
Indicate whether the following two processes are approximately Poisson. Explain your answer.
\begin{itemize}
\item[(a)] The times at which people arrive at the US airport immigration.
\item[(b)] The points in time at which an item is ordered at \url{amazon.com}.
\end{itemize}
\end{problem}

\begin{solution}
\begin{itemize}
\item[(a)] Not Poisson: The number of people arriving at the US airport immigrations are clustered and dependent on the flight schedule. This makes them \textbf{dependent} and therefore not suitable for Poisson distribution.
\item[(b)] This scenario may approximate a Poisson process under certain conditions. That is assuming a high volume and independence rate of orders. These assumptions could work to approximate this process as Amazon experiences a large volume of independent customer orders. While certain times (like holidays or sales events) might see spikes, during normal hours, individual orders from different users might be close to independent.
\end{itemize}
\end{solution}

\begin{problem}{Discrete Time Markov Chains}
Give an example of a discrete time Markov chain that is (explain your answer):
\begin{itemize}
    \item[(a)] Positive recurrent and its unique steady state distribution $\pi$ is equal to $\pi = \frac{(1, 2, 4, 8, 16)}{31}$.
    \item[(b)] Reducible with both transient and recurrent states.
\end{itemize}
\end{problem}

\begin{problem}{Discrete Time Markov Chains}
You toss a die repeatedly until the product of the last two outcomes is equal to $12$. What is the average number of times you toss your die?
\end{problem}

\begin{problem}{Continuous Time Markov Chains}
Consider the continuous time Markov chain of the queue length process of an M/M/1 queue. Construct the transition probability matrix of both the uniformized and embedded Markov chain. Do both these chains have the same steady state distribution whenever $\rho < 1$?
\end{problem}

\begin{problem}{Applications}
Assume Sue receives on average 50 emails per day and on average she has 12 unread emails in her mailbox. How many hours does it on average take before Sue has read an incoming mail? An email is marked as read as soon as Sue opens it.
\end{problem}

\begin{problem}{Applications}
Consider a tandem queueing network, that is, a Jackson network with $M = 2$ queues with $p_{1,2} = 1$, $p_{2,0} = 1$, $p_{0,1} = 1$, and $p_{0,2} = 0$. Explain how the speed $\mu_1$ of the first server affects the mean queue length and mean waiting time at the second queue.
\end{problem}

\begin{problem}{Bianchi Model}
Discuss the changes required to the 802.11 Bianchi model if we replace the uniform backoff time with a geometric backoff time. More specifically, indicate the changes required in the state space of the Markov chain used to determine the throughput.
\end{problem}