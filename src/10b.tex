\exam{2010b}
\begin{problem}{Bernoulli and Poisson Processes}
Argue that the minimum spanning tree (MST) of the basic binary tree algorithm does not change when we implement the following adjustment. A user who creates a new packet will send it in the next slot with probability $a$ and in the subsequent slot with probability $1 - a$.
\end{problem}

\begin{problem}{Branching Processes}
Suppose we use the basic binary tree algorithm on a channel with multiple reception capabilities. On such a channel, we can also correctly receive packets involved in a collision consisting of $k$ or fewer packets. How should we adjust the branching process from the course to determine the MST (as a function of $k$)? What should we do with $d$ when we increase $k$? How can we immediately see that $p = \frac{1}{2}$ is still an optimal choice?
\end{problem}

\begin{problem}{Markov Chains}
Consider an irreducible Markov chain with a finite number of states $n$. For which values of $n$ can the Markov chain be periodic with a period equal to 5? Explain your answer.
\end{problem}

\begin{problem}{Markov Chains}
Assume that people arrive at an attraction in an amusement park according to a Poisson process with rate $\lambda$ people per minute. Assume that a cart arrives every 90 seconds, accommodating (and requiring) 6 people. At which time points should we observe this system to obtain a Markov chain? What is the state space of the Markov chain, and formulate the transition matrix. For which values of $\lambda$ is your chain positive recurrent? What do you expect to happen to the average waiting time when a cart arrives every minute that accommodates only 4 people?
\end{problem}

\begin{problem}{Erlang C Formula}
Suppose we wish to generalize the Erlang C formula by also accounting for the impatience of waiting customers. We assume that the amount of patience of a waiting customer is exponentially distributed with a mean of $1/\theta$. We construct a Markov chain with the same state space as in the Erlang C formula, but we now observe the chain not only at arrival and service completion time points but also at the times when waiting customers lose their patience and leave. In short, when there is one or more waiting customers, there are three possibilities for the next event: an arrival ($\lambda$), a service completion ($\mu$), or a customer who loses their patience ($\theta$).
\begin{itemize}
    \item[(a)] Provide the transition matrix for this system; you do not need to determine the invariant vector.
    \item[(b)] When is this Markov chain positive recurrent (use Pakes)?
\end{itemize}
\end{problem}

\begin{problem}{Bianchi Model}
Assume that the maximum window size is 8 times greater than the minimum window size. In the course, a packet can be retransmitted an unlimited number of times; suppose we want to limit this number to 16. How should we adjust the Markov chain (without going into too much detail)?
\end{problem}