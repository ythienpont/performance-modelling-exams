\exam{2016}
\begin{problem}{Bernoulli and Poisson Process}
 Indicate whether the following statements are true or false. Explain your answer.
	\begin{itemize}
		\item[(a)] If some number was not part of the 6 lottery numbers during the last 10 weeks, it is more likely to be part of the six numbers of this week as all numbers appear equally often in the long run (due to the law of the large numbers),
		\item[(b)] Points in time at which queries arrive at a DNS server, which translates domain names in IP addresses, is fairly well approximated by a Poisson process,
		\item[(c)] the time epochs at which students log on to Blackboard for a specific course.
	\end{itemize}
\end{problem}

\begin{problem}{Branching Processes}
Determine the extinction probabilities $q_1$, $q_2$, and $q_3$ for the
following multi-type branching process, where $p^{(s)}_{i,j,k}$ is the probability that a type $s$ individual
has $i$ children of type $1$, $j$ of type $2$, and $k$ of type $3$:
\begin{itemize}
    \item $p^{(1)}_{1,0,4} = 0.5$, $p^{(1)}_{0,1,3} = 0.25$, and $p^{(1)}_{0,2,1} = 0.25$.
    \item $p^{(2)}_{0,0,3} = 0.2$, $p^{(2)}_{0,1,2} = 0.4$, and $p^{(2)}_{0,2,17645} = 0.4$.
    \item $p^{(3)}_{0,0,k} = \binom{16}{k} \left( \frac{1}{20} \right)^k \left(1 - \frac{1}{20}\right)^{16 - k}$, for $0 \leq k \leq 16$.
\end{itemize}
\end{problem}

\begin{problem}{Markov Chains}
Consider an irreducible, finite Markov chain with transition matrix
$P = P_w + P_b$, with $P_w$ and $P_b$ substochastic as in Exercise 17 in the course notes. If
we only observe this Markov chain after the white transitions, we obtain a new Markov
chain.
\begin{itemize}
    \item[(a)] Show by means of an example that the new Markov chain is not necessarily irreducible.
    \item[(b)] Is it possible that this new Markov chain contains two (or more) closed communicating classes?
\end{itemize}
\end{problem}

\begin{problem}{Bianchi Model}
How would you proceed to determine the saturation throughput of
the 802.11 DCF function if we replace the binary exponential back-off algorithm with
a simple ALOHA scheme (i.e., uniform back-off between $0$ and $W$)? What are
the required changes to the Bianchi model? Give an explicit expression for the failure
probability $p$.
\end{problem}

\begin{problem}{Markov Chains}
Suppose customers arrive at an ATM machine according to a Poisson
process with rate $\lambda$, and their transactions have a constant duration (e.g., 1 minute).
Set up a Markov chain to determine the probability that a customer finds exactly $i$
customers queued at the ATM machine at the time of their arrival. [Hint: the probability
that there are $i$ customers queued at an arrival time is identical to the probability that
there are $i$ customers queued at a departure time (i.e., when a customer completes their
transaction).]
\end{problem}

\begin{problem}{Bernoulli and Poisson Processes}
Suppose we split a Poisson process in a probabilistic manner into process $A$ and $B$ and that process $A$ is again split in a probabilistic
manner into processes $A_1$ and $A_2$. What is the mean arrival rate of the superposition of
processes $A_2$ and $B$? Is this superposition also a Poisson process?
\end{problem}

\begin{problem}{Erlang B Formula}
Suppose that a telecom operator has both premium and regular
customers, and whenever $C_p$ or more of the available $C$ lines are occupied (with $C_p < C$),
only premium customer calls are accepted. Adapt the Markov chain in the course notes
to incorporate this policy (it is not necessary to determine an expression for the steady
state probabilities). How would you determine the blocking probabilities from the steady
state probability vector (detailed formulas are not required)?
\end{problem}
