\exam{2019}
\begin{problem}{Poisson Process}
Consider a Poisson process with rate $ \lambda$. Assume $n$ arrivals occurred in between time $0$ and $t$. Give an expression for the probability that no arrivals occur between time $t/3$ and $2t$? Explain how you obtained your expression.
\end{problem}

\begin{solution}
  Arrivals are independent. Even though there were $n$ arrivals between $0$ and $t$, this doesn't influence the probability of having no arrivals between $t/3$ and $2t$.

  \[
    P[I_n \leq t] = 1 - \exp(-\lambda t) \Rightarrow P[I_n > t] = 1 - (1 - \exp(-\lambda t)) = \exp(-\lambda t)
  \]

  We find our $t$ as $2t -t/3 = \frac{5t}{3}$. Our final expression is $\exp(-\lambda \frac{5t}{3})$.
\end{solution}

\begin{problem}{Discrete Time Markov Chains}
What is the least number of non-zero entries in the transition probability matrix $P$ of a discrete time irreducible aperiodic Markov chain with $n > 2$ states? Does your answer change if we demand that the chain has period 2? Explain your answers.
\end{problem}

\begin{solution}
  For an irreducible aperiodic Markov chain each state we need a single SCC. We need one transition per node to make cycle, and one more for a self loop to ensure aperiodicity, so $n+1$.

  For an irreducible chain with period 2, it differs for even or odd states. For an even amount of states we create a cycle and then a single transition in the reverse direction, since the period is defined as the greatest common divisor of all the time epochs at which the system can be in state $i$, a cycle with even nodes and a possibility to increase the cycle with any multiple of two gives us a period of 2. So $n+1$ non-zero entries.
  For an odd amount of states, we need to make a chain with two transitions per state, $2(n-1)$ non-zero entries. (Think of a BD chain)

  % TODO: Add a drawing, I don't think this is too hard to imagine however.
\end{solution}

\begin{problem}{Continuous Time Markov Chains}
Consider a continuous time Markov chain on the state space $S = \{(i, j)|i \geq 0, j \geq 0\}$ and assume the transition rates are such that
\[
q_{(i,j),(i+1,j-1)} = 2, \text{ for } j > 0, \quad
q_{(i,0),(0,i-1)} = 2, \text{ for } i > 0,
\]
\[
q_{(i,j),(i-1,j+1)} = 1, \text{ for } i > 0, \quad
q_{(0,j),(j+1,0)} = 1.
\]
Prove that this Markov chain is positive recurrent.
\end{problem}

\begin{problem}{You've Got Mail}
Bob receives both regular mail and spam mail. An incoming
spam mail is detected as spam with probability $p$ and moved to Bob’s spam folder.
Undetected spam mails are collected in Bob’s regular mail folder together with all of his
regular mails. Bob has on average 4 unopened mails in his regular mail folder and 80
mails in his spam folder. Bob cleans his spam folder every 5 days and opens all of his
unopened regular mail every 3 hours (on average). Which fraction $p$ of Bob’s incoming
spam mail is certainly detected as spam?
\end{problem}

\begin{problem}{Queueing Theory}
Consider a queue with Poisson arrivals with rate $1$ and exponential
service times with mean $1/2$. Assume jobs are served first-come-first-serve and at most
$N$ jobs can wait in the waiting room. Incoming jobs that find the waiting room full
are lost (i.e., immediately leave). Show that the loss probability in this system equals
$\frac{1}{2N + 1 - 1}$.
\end{problem}

\begin{problem}{Queueing Networks}
Consider a system consisting of 3 queues (labeled 1, 2, and 3)
and assume we have $N$ jobs in the system. A job requires an exponential amount of
service with mean $1/\mu_i$ in queue $i$ and moves to queue $(i \mod 3) + 1$ afterwards. Thus
we have a fixed number of $N$ jobs in the system. The service discipline is first-come-first-serve in each queue. Show that the probability to have $i$ jobs in queue 1 and $j$ jobs
in queue 2 is given by the following product form
\[
\frac{\frac{1}{\mu_1^i \mu_2^j \mu_3^{N-i-j}}}{\sum_{i=0}^{N} \sum_{j=0}^{N-i} \frac{1}{\mu_1^i \mu_2^j \mu_3^{N-i-j}}}.
\]
\end{problem}

\begin{problem}{Bianchi Model}
Discuss the changes required to the 802.11 Bianchi model if we replace
the uniform backoff time by some other distribution on $\{0, \ldots, W_i -1\}$, for $i = 1, \ldots, m$.
More specifically, indicate the changes required in the Markov chain used to determine
the throughput. Why does the uniform distribution seem like a sensible choice?
\end{problem}
